\documentclass[11]{article}
\usepackage[utf8]{inputenc}
\usepackage[norsk]{babel}
\usepackage{amsmath}
\usepackage{gensymb}
\usepackage{parskip}
\usepackage{booktabs}
\usepackage{adjustbox}
\usepackage{tikz}
\usepackage{graphicx}
\usepackage{array}
\usepackage{helvet}
\usepackage{textcomp}
\renewcommand{\familydefault}{\sfdefault}
\renewcommand{\baselinestretch}{1.15} 



\begin{document}
\renewcommand{\thefootnote}{\roman{footnote}}
\title{}
\author{}
\date{12.09.2019}

\begin{titlepage}
   \begin{center}
 
        \textbf{\huge TIØ4252 Teknologiledelse}\\
        \vspace{4cm}
        \textbf{\huge Øving 2}\\
        \vspace{0.5cm}
        \textbf{\small Gruppe 1, parallell 2}
        
        \vspace{5cm}
        
        \textbf{\small\\*Christopher Janjua\\* }
        \textbf{\small\\*Erlend Marius Ommundsen\\* }
        \textbf{\small\\*Peder Solheim\\* }
        \textbf{\small\\*Runar Sæther\\* }
        \textbf{\small\\*Sivert Laukli\\* }
        
        \vspace{1cm}
 
        \includegraphics[width=0.4\textwidth, draft=false]{Vedlegg/ntnuLogo.png}
 
        03.10.2019
 
   \end{center}
\end{titlepage}

\newpage



\newpage
\tableofcontents 

\newpage
\section{Organisasjonsstruktur}

INTRODUKSJON TIL OPPGAVE 1


\newpage
\subsection{Teori}

Samordning er et sentralt begrep som er mye brukt når vi skal diskutere hvordan et selskap opererer.
I arbeidsinstruksen til en ansatt kan vi finne noe av samordningen til et selskap. Dette kan være en del i en større prosess som må utføres før den neste kan utøve sin del av arbeidet, som for eksempel at før en vei kan oppmerkes så må asfalt være lagt.
Dette er samordning som er oppgitt i arbeidsbeskrivelsen, men det er nødvendig med mer samordning enn dette. 
Samordningen er viktigst når det kommer til uregelmessige hendelser som må tas stilling til, eksempelvis krisesituasjoner.
Dette kaller vi samordningsmekanismer. Samordningsmekanismer deles inn i to hovedtyper. Vertikal- og horisontal-samordning $^1$.

De vertikal samordningsmekanismene skal håndtere behovet for samordning mellom de ulike nivåene i et selskaps hierarki. De skal sørge for at informasjon mellom lederne og ansatte flyter fritt, ved å bruke ulike aktiviteter som støtter opp dette. Eksempler på slike aktiviteter er medarbeidersamtaler, forslagskasser og rapporter som kommer fra ansatte.

De horisontale samordningsmekanismene skiller seg fra de vertikale ved å fokusere på koordinering mellom de på samme hierarkiske nivået. Her er det som mål å sørge for at dobbelt arbeid ikke oppstår, ved ha god fordeling og oversikt over hverandres arbeid, og dele kompetanse med hverandre. Noen eksempler på dette kan være kollegaveiledning og møter. 

Fra Henry Mintzberg sin modell$^1$ blir det beskrevet om fem ulike å måter å bedrive samordning på. Den første nevnte er "Direkte kontroll". Dette foregår ved at lederne aktivt tar del i arbeidsprosessen. Dette gir lederne mulighet til å gi direkte tilbakemelding til sine ansatte. Tilbakemeldingen er umiddelbar og kan gi resultater uten tidsforsinkelser. Fordelen med at lederne tar en slik rolle er at de er i stand til å kvalitetssikre egen produksjon og kvantitet som blir produsert. Dette er på annen side veldig krevende for lederne å gjennomføre. Det kreves at lederne har  høy kompetanse og godt faglig grunnlag innen fagfeltene de observerer for å kunne gi tilbakemeldinger. Direkte kontroll blir i dag ikke brukt like mye slik som før, men blir brukt i sammenheng med prosesser som er tidskritisk, eksempelvis brann og militær ledelse.

Et annet eksempel i modellen er "Prosesskontroll". Her er det om å optimalisere arbeidsprosessene på best mulig måte. Maskiner er mye brukt i denne sammenhengen da prosesskontroll ofte handler om å redusere slakk. Maskiner er forutsigbare når det kommer til tidsbruk. Et selskap som bruker denne typen kontroll trenger begrenset kompetanse, da den ansatte skal gjøre en veldig begrenset og konkret oppgave, som en prosessingeniør har delt opp.

Resultatkontroll er annen kontroll type hvor lederne bare ser på resultatet av arbeidet som er blitt utført. Utifra dette vurderes det om arbeiderne gjør en god jobb. Dette er som regel en enkel kontrolltype å gjennomføre, og krever få ressurser. Det gir muligheten til de ansatte å jobbe på sin egen måte, bedre utnyttet kompetanse. Dette vil igjen resultere til bedre motivasjon når de ansatte får brukt sin tilegnede kompetanse. Ulempen med denne typen kontroll er at det er mindre oversiktlig for lederne, og det er tillit til de ansatte om at arbeid blir utført. Det åpner ikke for like mye dialog mellom den ansatte og ledere om potensiell prosessforbedring. 

\subsection{Nordvest-bygg AS}

Casen om Nordvest-bygg AS tok for seg et byggeselskap med opprinnelse i Ålesund. Vi så på selskapets ekspansjon fra å drive med renovering, til å utvide sitt kompetansegrunnlag ved å påta seg bygging av nybygg.

Nordvest-bygg AS opplevde en voldsom ekspansjon over en kort tidsperiode. Dette førte til den vertikale samordningen begynte å slå sprekker. En første omorganisering måtte til, da bedriften hadde problemer med å tilpasse organisasjonen med det stadig større antallet ansatte. Allerede her tas det valg som får konsekvenser for bedriften frem i tid. Hver av de fusjonerte avdelingene fungerte nå som autonome enheter som styrte seg selv og som hadde egne avtaler. En slik oppdeling vil gi ledelsen mindre oversikt over hva som faktisk foregår innad i avdelingene, og sier mer om at ledelsen ønsker resultater. Man ser allerede her at sterkere vertikale samordningsmekanismer trengs.

Etter ekspansjonen gikk bedriften inn i en konsolideringsfase, og da spesielt i forbindelse med arbeidsmiljøet. En arbeidsmiljøkomité blir dannet, som besøker byggeplasser og regionskontor for å ha samtaler med ansatte. Det kommer frem at de hadde samtaler med over halvparten av alle ansatte. Disse samtalene er et eksempel på fornuftig bruk av vertikal samordning, da det knyttes bånd mellom personene på de forskjellige nivåene i hierarkiet, og ledelsen får et bedre innblikk i hva som foregår på de lavere nivåene.

Etter konsolideringsfasen solgte eierne seg ut til et nasjonalt selskap med store summer til grunn, og bedriften gikk helt bort fra det den en gang var. En ny eier kom inn i avdelingen som tidligere var Nordvest-bygg og etter en ny omorganisering begynte ansatte å sluntre unna prosjekter med mye reising. Dette er en direkte negativ konsekvens av resultatkontroll, hvor ledelsen i dette tilfellet fokuserer på det ferdige resultatet i motsetning til veien dit. Unnasluntringen førte til sparking av ansatte og prosjektledere, og skapte en nærmest krigstilstand mellom leddene vertikalt i hierarkiet. Altså er dette et tydelig eksempel på problematisk bruk av vertikal samordning.

\subsection{Stol AS}
I casen om Stol AS beskrives utviklingen av en møbelfabrikk på Røros, som startet i det små med utviklingen av stoler basert på en fersk patent på et stivere laminat, og utviklet seg i korte trekk ved å innlemme lokale produsenter av råmateriale fra området rundt til å bli en aktør på flere segmenter av møbelmarkedet.

Parallelt med ekspansjonen av selskapet beskrives også 'voksesmertene' for organisasjonsstrukturen. De drastiske endringene skjedde over kort tid, og ledelsen i selskapet måtte stadig gjøre endringer for å optimalisere produksjonen og overleve.

Da gründeren, Mikal Hansen, døde i 1967, var bedriften midt i en utviklingsprosess av en ny teknisk løsning for laminatet brukt i stolene. Stol AS mistet dermed pådriveren i selskapet, som de erstattet med den ferdigutdannede maskiningeniøren Torstein Varhei. En av de første endringene Varhei innførte var jobbrotasjon på produksjonslinjene. Dette betydde at alle ansatte måtte lære seg tre arbeidsstasjoner. Som et resultat av denne endringen, fikk de ansatte en mer variert arbeidshverdag og det ble lettere å finne en midlertidig erstatter til syke ansatte. Jobbrotasjonen og en endring på antall ansatte per avdeling førte til slutt til økt kvalitet i sluttproduktet. Dette er et eksempel både på god horisontal samordning og prosesskontroll.

\subsection{Vest-Regnskap AB}
Vest-Regnskap AB ble ifølge casen stiftet som et tradisjonelt regnskapsbyrå i 1976, men ble, som følge av Monika Halléns innsats, et firma som ga unge foreldre en mulighet til å få jobberfaring samtidig som det ga dem tid til å være hjemme med barna. Dette klarte de ved å være tuftet på deltidsstillinger. 

Mot slutten av casen, etter fusjonen med Roger Mostrøms konsulentselskap, oppstod det opp komplikasjoner mellom tre parter med toppstillinger i firmaet. Den første parten var personalansvarlig Mette Zetterlund, den andre var daglig leder Mats Johansson og den tredje var lederen av Malmø-avdelingen, Roger Mostrøm. Da Roger forkastet Mettes forslag om å ta inn deltidsarbeidende i Malmø, på samme måte som den i andre avdelingen, gikk Mette til Mats og fikk ham til å snakke med Roger. Da Mette spurte Mats om saken i ettertid, fikk hun ikke noe klart svar. Til slutt fikk Mette et brev på pulten sin som sa at personalansvaret for Malmø-avdelingen var nå kun Roger sitt, og all kommunikasjon med Roger skulle skje gjennom Mats. Denne kommunikasjonsprosessen er et eksempel på dårlig horisontal samordning. Ikke bare følger det en forverring av forholdet mellom Mette, Mats og Roger, det fører til at kommunikasjonen vil gå tregere mellom partene på grunn av den ekstra forsinkelsen, nemlig Mats.  

\newpage
\section{Ledelse og motivering av ansatte.}

Vis ved hjelp av tre konkrete eksempler fra
casesamlingen hvordan tre ulike motivasjonsteorier er brukt på en god måte. Hvert eksempel
skal vise en motivasjonsteori i bruk, og dere må kort beskrive teorien først og deretter vise
hvordan teorien er satt ut i praksis. Eksemplene kan komme fra samme case eller fra ulike
case.

\subsection{Teori}


Herzberg tofaktorteori hevder at det er to grupper av faktorer som påvirker motivasjon. De er hygienefaktorer og motivasjonsfaktorer. Motivasjonsfaktorer kan øke trivsel, I motsetning til hygienefaktorer som kun kan bidra til å hindre mistrivsel. Hygienefaktorer må være tilstede i en tilstrekkelig mengde i et arbeidssystem for å unngå at ansatte blir fraværende. Med bruken av ordet hygiene menes at det er noe som må opprettholdes kontinuerlig. Noen hygienefaktorer er mellommenneskelige relasjoner og arbeidsbetingelser, personalpolitikk osv. Noen motivasjonsfaktorer er lønn, personlig vekst, ansvar for eget arbeid, prestasjoner osv. 

\subsection{Nordvest-Bygg AS}

Etter å ha brukt mye tid på strategier som resulterte i vekst av Norvest Bygg As, vendte de oppmerksomheten mot arbeidsmiljøet i bedriften. De visste at vekst og endringer hadde skjedd raskt og at det derfor var et behov for å tilpasse bedriften tilsvarende. En trivselundersøkelse viste at de elementene som skapte stor påkjenning for ansatte var reisingen og ukependlingen mellom hjem og byggeplass. Mange av de ansatte var bekymret for ulykker på grunn av sikring av stillaser, heiser og kraner. Positive ting som kom frem av trivselundersøkelsen var at de ansatte av fornøyd med lønning og arbeidsoppgaver. Bedriften opprettet en arbeidsmiljøkomite som la frem rapport basert på antall reisedager, pendlerdager, ulykkesstatistikk, og samtaler med opp imot 200 ansatte. 

«17 punkter ble nevnt, og de tre viktigste var en oppgradering av det fysiske arbeidsmiljøet (bedre møbler og utstyr på brakkebyene der pendlerne bodde i arbeidsuka), muligheter for besøk fra familien (delvis betalt av bedriften) og bedre rotasjon blant arbeidsstokken.»

Dette resulterer i at Nordvest-Bygg AS installerer bedre møbler og utstyr i brakkebyene hvor ansatte bor imens de pendler. De setter opp muligheter for besøk av familien og bedre rotasjon blant hvem som reiser når og hvor de reiser. De reduserer antall ulykker ved hjelp av økte sikkerhetsrutiner. Dette er flere punkter som går under gruppen hygienefaktorer i Herzberg tofaktorteori. Bedre møbler og utstyr i brakkebyene samt mindre reisedøgn og redusert antall ulykker kan minke mistrivsel da arbeidsbetingelser er en hygienefaktor. Økte muligheter for besøk fra familie delvis på bekostning av bedrift, kan minke mistrivsel da mellommenneskelige relasjoner er en hygienefaktor. Så i henhold til Herzberg tofaktorteori var viktige hygienefaktorer fraværende i arbeidsmiljøet i bedriften, noe som skapte mistrivsel hos de ansatte. De ansatte var allerede fornøyd med arbeidsoppgavene og lønnen. Dette er motiavsjonsfaktorer som ikke var fraværende. For at disse motivasjonsfaktorene skal få best uttelling så må hygienefaktorer være på plass og opprettholdt kontinuerlig. Noe som Norvest Bygg gjorde på en god måte. Dette ser vi ettersom «... ting roet seg i bedriften, og siden tidene var gode kunne Nordvest-Bygg fortsette som før i noen år frem til 1997.» Endringer som skjedde i 1997 var urelatert til motivasjon av ansatte. 

\subsection{Vest-Regnskap AB}

\newpage
\section{Teamarbeid}

\newpage
\section{Den norske modellen}


\newpage
\section{Referanser og bibliografi}

1.  Torvatn, Rolfsen, Heggernes, Sørheim. \textit{Teknologiledelse - for ingeniørstudenter} (s. 111-117). Fagbokforlaget, 2016. \\
2. Herzberg, Frederick, One More Time: How Do You Motivate Employees?, Harvard Business Review, 1987 \\
\end{document}
